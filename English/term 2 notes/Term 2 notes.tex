\documentclass[11pt]{scrartcl}
\usepackage[left=1.1in,right=1.1in]{geometry}
\usepackage{sprint}
\usepackage[toc]{glossaries}
\newglossary[alg]{amanda}{ald}{adn}{Glossary}
\newglossary[blg]{animals}{bld}{bdn}{Glossary}


\rhead{\textsf{ENGLISH NOTES}} %Rightside
\lhead{\textbf{Pragyan Pranay}} %Leftside

\makeglossaries

\loadglsentries{Glossary amanda.tex}
\loadglsentries{Glossary-Animals.tex}


\begin{document}
    \title{English Term-2 Notes}
    \author{Pragyan Pranay}
    \date{24 January, 2022}
 \maketitle

\tableofcontents


\part{Poems}
\section{Amanda}
This poem is all about a child like me ranting about how 
parents are always acting around boss and bullying us 
every time about not studying and commanding and controlling 
us in such a manner that we feel so irritated and lost that 
we get the inspiration to write this poem. 

This poem is written by \textbf{Robin Klein} and has a total
of 7 paragraphs or whatever. The paragraphs alter between 
imagination of Amanda and her mother's scoldings.

\vspace{4mm}

\noindent \textbf{Paragraph 1:} In the first paragraph, we 
see Amanda's
mother telling her \textit{not to }
\begin{itemize}
    \item bite her nails
    \item \gls{hunch} her shoulders
    \item to stop \gls{slouching} and sit straight up
\end{itemize}

\noindent \textbf{Paragraph 2:} Here she is imagining that 
she is a mermaid who is in a \gls{emerald} coloured \gls{languid} sea 
who is moving around \gls{blissfully}.

\vspace{4mm}

\noindent \textbf{Paragraph 3:} This is time for homework 
which is one of the most annoying things to a child. Later 
she asked her if she had cleaned her room and cleaned her 
shoes or not.

\vspace{4mm}

\noindent \textbf{Paragraph 4:} IMAGINATION! This one is 
quite interesting. She said here that she is an orphan which 
is clearly not true. She is trying to make patterns in the 
soft dust with her \textit{\gls{hushed}, bare feet}. Another 
thing to notice is that she is thinking the opposite of her 
mother. At one side, her mother wants her to keep herself tidy 
and well-arranged whereas Amanda wants to play in the dust, 
without any worries.She also compliments silence as 
``golden" and freedom as ``sweet''. Very ordinary stuff after 
the constant \textit{nagging} by parents.

\vspace{4mm}

\noindent \textbf{Paragraph 5:} Don't eat chocolate because 
of cavities and more importantly as a growing child, you do 
have to face acne. Also, always look at your parents when 
they are talking to you.

\vspace{4mm}

\noindent \textbf{Paragraph 6:} She is in another world 
right now. She says that she is Rapunzel and she does not 
about all the worldly affairs. She finds her life \gls{tranquil} 
in her own dream world. She also denies Rapunzel's idea 
of letting her hair down and escaping the tower and values 
the calmness in the tower.

\vspace{4mm}

\noindent \textbf{Paragraph 7:} Mother starts again! Her 
mother says that Amanda was moody and said her to 
stop \gls{sulking}. She said that the way she was behaving, people 
would think that Amanda was \gls{nagged} by her mother. Poor little 
Amanda couldn't even say that it was indeed the case!

\printglossary[type=amanda]

\section{Animals}
This chapter is about a depressed guy who is too weak to live 
in this world and says that he wants to live with animals, where 
there is no competition, and everything is peaceful. He will 
continuously say that animals are very good and they do not 
complaint about their situations no matter what happens. He 
feels very comfortable with those animals who stay as who they 
are without acting and giving false impressions. This basically 
means that he is fed up with life and wants to go to Himalayas 
and leave all the false people in the world.

\vspace{3mm}

This poem is written by Walt Whitman who is said to be a very 
nice and well-known poet. This poem is of about 4 paragraphs 
where the last paragraph is short and concludes the poem. I 
do oppose the poet at several points, but I do get how he feels.
Let us begin.

\vspace{4mm}

\noindent \textbf{Paragraph 1:} He introduces us the notion 
of living with animals and starts to appreciate them. He tells 
everyone that animals are so \gls{placid} to live with. They are 
also self contain'd which means that they are happy and never 
interfere with other's life. He keeps gazing at those lovely 
animals admiring the animals.

\vspace{3mm}
\printglossary[type=animals]
\end{document}