\documentclass[11pt]{scrartcl}
\usepackage[left=1.1in,right=1.1in]{geometry}
\usepackage{sprint}

\rhead{\textsf{History}} %Rightside
%\lhead{\textbf{Pragyan Pranay} (\today)} %Leftside
\begin{document}
    \title{History Notes}
    \author{Pragyan Pranay}
 \maketitle

\tableofcontents
\section{Nationalism in India}
\begin{question}
    How did World War 1 helped in the growth of nationalism 
    in India?
\end{question}
\begin{ans}
    \begin{enumerate}
        \item War led to increase in taxes, custom duties and 
              other commodities. It also led to introduction 
              of \textit{income taxes.}
        \item Villages were called upon for supplying 
              soldiers for war and this \textit{forced 
              recruitment} angered the people.
        \item Between the years 1918-1919 and 1920-21, there 
              was major crop failure which led to acute 
              shortage of food.
        \item To add to the sufferings, \textit{influenza} 
              epidemic also occurred at the time of these 
              famines.
        \item According to the census of 1920-21, 
              approximately 13 million people died as a 
              result of famines and epidemic.
    \end{enumerate}
    People expected that their sufferings would end after 
    the war, but that didn't happen.    
\end{ans}

\begin{question}
    What do you mean by satyagraha and why did Gandhi ji 
    appeal to follow it?
\end{question}
\begin{ans}
    \begin{enumerate}
        \item The idea of satyagraha emphasized on on the 
              power of truth and the need to search for truth.
        \item Satyagraha seeked victory in battle without  
              vengeance or being aggressive.
        \item The idea was to persuade the oppressors to 
              see the truth. 
    \end{enumerate}
    
    Gandhi ji appealed to follow this because:
    \begin{enumerate}
        \item He got recent success in South Africa.
        \item He thought that non-violence could unite all 
              Indians.
    \end{enumerate}
\end{ans}
\begin{question}
    Explain the three earliest satyagrahas launched by 
    Mahatma Gandhi in India.
\end{question}
\begin{ans}
There were three moments launched by Mahatma Gandhi
    \begin{enumerate}
        \item He first launched a movement in Champaran(Bihar) 
              in the year 1917 against the oppressive plantation 
              system imposed by British.
        \item Gandhi ji later went to Kheda(a district in Gujrat)
              to support peasants in the same year 1917. The 
              peasants were drastically affected by crop failure 
              and a plague pandemic. Due to this, they were unable 
              to pay the revenue and demanded the collection to 
              be relaxed.
        \item In 1918, he went to Ahemdabad to organize a 
             \textit{Satyagraha Movement} with the cotton mill 
              workers of that area.
    \end{enumerate}
\end{ans}
\begin{question}
      The growth of modern nationalism is intimately connected 
      to the anti colonial movement in india.  Explain.
\end{question}
\begin{ans}
      \begin{enumerate}
            \item People began sensing nationalism in the 
                  process of their struggle with colonialism.
            \item The sense of being oppressed under 
                  colonialism provide a share bond that 
                  tied many different groups together.
            \item But each group felt the effects of 
                  colonialism differently , their experiences 
                  varied and the notion of freedom were not 
                  always the same.
      \end{enumerate}
\end{ans}
\begin{exercise}
      Write a short note on Khilafat issue.
\end{exercise}

\begin{question}
      Why did Gandhi ji chose salt as symbol of resistance?
\end{question}
\begin{ans}
      Gandhi Ji used salt as a symbol of resistance because:
      \begin{enumerate}
            \item Salt was an effective symbol of 
                  resistance as it was a revolt against a 
                  commodity , salt which was consumed by 
                  both rich and poor
            \item The governments monopoly over the 
                  production of salt and the tax on salt was 
                  a really oppressive administrative move.
            \item By denying salt law and manufacturing salt 
                  against government permission Gandhiji set 
                  a forth example of how the oppressor could 
                  be defeated with non violence.
      \end{enumerate}
\end{ans}
\begin{question}
      Differentiate Non-Cooperation and Civil Disobedience 
      movement.
\end{question}
\begin{ans}
      
\end{ans}
\begin{question}
      What measures were taken to uplift the condition of the 
      "untouchables"?
\end{question}
\begin{ans}
      He did this because he thought that India won't get 
      freedom until 100 years unless \emph{dalits} were not 
      a part of movement.
      \begin{enumerate}
            \item He called "untouchables" as \textit{harijans} 
                  or the children of god and organized satyagraha 
                  to provide them access to temples and public 
                  wells, tanks and road.
            \item He himself cleaned toiler to help in the 
                  upliftment of the untouchables. 
            \item He persuaded upper castes to change their 
                  heart and give up on the ``sin of untouchability'' 
                  and accept them as a people of our own country.
      \end{enumerate}
\end{ans}
\begin{question}
      Describe Poona Pact 1932.
\end{question}
\begin{question}
      Why did the muslims show lukewarm response towards Civil 
      Disobedience Movement?
\end{question}
\begin{ans}
      \begin{enumerate}
            \item After the calling of first Non-Cooperation 
                  muslims felt alienated from Congress.
            \item In mid-1920s, the Congress gave more attention 
                  to hindu religious nationalist groups like 
                  Hindu Mahasabha.
            \item As relations between Hindu and Muslim worsened, 
                  each community organized their own religious 
                  professions.
            \item The Muslim League under the leadership of 
                  Muhammad Ali Jinnah demanded separate electorates
                  in elections.
            \item Later Jinnah said that they would leave the 
                  the demand of separate electorates if they 
                  were given reserved seats in the election.
                  But Congress completely rejected the the 
                  muslims to have reserved seats in elections.
      \end{enumerate}
\end{ans}
\begin{question}
      Explain the effects of Non-Cooperation on the economic 
      front.
\end{question}
\begin{ans}
      The effects were as follows
      \begin{enumerate}
            \item Foreign cloth boycotted, liquor shops were 
                  picketed.
            \item Value of imported foreign cloth dropped 
                  from 102cr to 57cr during 1921-22.
            \item Traders refused to trade foreign goods and 
                  refused to finance foreign goods.
      \end{enumerate}
\end{ans}
\end{document}
