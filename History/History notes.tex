\documentclass[11pt]{scrartcl}
\usepackage[left=1.1in,right=1.1in]{geometry}
\usepackage{sprint}

\rhead{\textsf{History}} %Rightside
%\lhead{\textbf{Pragyan Pranay} (\today)} %Leftside
\begin{document}
    \title{History Notes}
    \author{Pragyan Pranay}
 \maketitle

\tableofcontents
\section{Nationalism in India}
\begin{question}
    How did World War 1 helped in the growth of nationalism 
    in India?
\end{question}
\begin{ans}
    \begin{enumerate}
        \item War led to increase in taxes, custom duties and 
              other commodities. It also led to introduction 
              of \textit{income taxes.}
        \item Villages were called upon for supplying 
              soldiers for war and this \textit{forced 
              recuritment} angered the people.
        \item Between the years 1918-1919 and 1920-21, there 
              was major crop failure which led to acute 
              shortage of food.
        \item To add to the sufferings, \textit{influenza} 
              epidemic also occured at the time of these 
              famies.
        \item According to the census of 1920-21, 
              approximately 13 million people died as a 
              result of famines and epidemic.
    \end{enumerate}
    People expected that their sufferings would end after 
    the war, but that didn't happen.    
\end{ans}

\begin{question}
    What do you mean by satyagraha and why did Gandhi ji 
    appeal to follow it?
\end{question}
\begin{ans}
    \begin{enumerate}
        \item The idea of satyagraha emphasized on on the 
              power of truth and the need to search for truth.
        \item Satyagraha seeked victory in battle without  
              vengance or being aggressive.
        \item The idea was to persuade the oppressors to 
              see the truth. 
    \end{enumerate}
    
    Gandhi ji appealed to follow this because:
    \begin{enumerate}
        \item He got recent success in South Africa.
        \item He thought that non-violence could unite all 
              Indians.
    \end{enumerate}
\end{ans}
\begin{question}
    Explain the three earliest satyagrahas launched by 
    Mahatama Gandhi in India.
\end{question}
\begin{ans}
There were three moments launched by Mahatma Gandhi
    \begin{enumerate}
        \item He first launched a movement in Champaran(Bihar) 
              in the year 1917 against the oppressive plantation 
              system imposed by British.
        \item Gandhi ji later went to Kheda(a district in Gujrat)
              to support peasants in the same year 1917. The 
              peasants were drastically affected by crop failure 
              and a plague pandemic. Due to this, they were unable 
              to pay the revenue and demanded the collection to 
              be relaxed.
        \item In 1918, he went to Ahemdabad to organise a 
             \textit{Satyagraha Movement} with the cotton mill 
              workers of that area.
    \end{enumerate}
\end{ans}
\end{document}