\documentclass[11pt]{scrartcl}
\usepackage[left=1.1in,right=1.1in]{geometry}
\usepackage{sprint}

\rhead{Geography} %RightSide
%\lhead{Pragyan Pranay (\today)} %Leftside

\begin{document}
\title{Geography Notes}
\author{Pragyan Pranay}
\maketitle

\tableofcontents

\section{Manufacturing Industries}
\begin{question}
    Why is Manufacturing Sector considered as the ``backbone 
    of development''?
\end{question}
\begin{ans}
    This is because:
    \begin{enumerate}
        \item Manufacturing industries play an important role 
              in modernising agriculture , reducing dependence 
              of people in agricultural income and creating 
              jobs in secondary and Tertiary sector.
        \item Industrial development also eradicates unemployment 
              and poverty from our country.
        \item This was also the reason behind public sector 
              industries and joint sector ventures in India. 
              It also reduced regional disparities by 
              establishing industries in tribal and backward 
              classes.
        \item Export of manufactured goods expands trade and 
              commerce and brings in foreign trade.
        \item India's prosperity lies in increasing and 
              diversifying its manufacturing industries as 
              quickly as possible. Countries which transform 
              raw materials into variety of furnished goods 
              is considered prosperous.
    \end{enumerate}
\end{ans}
\begin{question}
    Agriculture and Industry are not exclusive. Justify the 
    statement.
\end{question}
\end{document}